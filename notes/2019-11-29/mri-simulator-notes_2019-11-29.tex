\documentclass[dvipsnames]{beamer}
\input{mybeamerdefs}

\title{MRI Simulator Notes}
\author{Liam}
\date{\today}

\begin{document}

\begin{frame}
\maketitle
\end{frame}

\begin{frame}{Tasks}
\begin{itemize}
\item I looked into distributed computing using Python.
\item I looked into random number generation.
\item I implemented excitation pulses in our simulator.
\end{itemize}
\end{frame}

\section{Distributed computing using Python}

\begin{frame}
\begin{itemize}
\item The module \href{https://github.com/ray-project/ray}{Ray} seems well-suited to our application.
\item Ray allows you to modify serial code for distributed computing (using a cluster or in the cloud) by simply adding decorators to existing functions and classes (\href{https://towardsdatascience.com/modern-parallel-and-distributed-python-a-quick-tutorial-on-ray-99f8d70369b8}{reference}).
\item See the reference page \href{https://ray.readthedocs.io/en/latest/}{here}.
\end{itemize}
\end{frame}

\section{Random number generation}

\begin{frame}
\begin{itemize}
\item If we want fast random number generation, I think we are better off with a software pseudorandom number generator.
\item Hardware random number generators create a sequence of numbers that is more truly random; however, software pseudorandom number generators have a higher output rate.
\item This \href{https://en.wikipedia.org/wiki/Hardware_random_number_generator}{Wikipedia entry} says ``hardware random number generators generally produce only a limited number of random bits per second."
\item \href{https://www.iacr.org/archive/ches2013/80860154/80860154.pdf}{Cherkaoui et. al. 2013} report on a hardware random number generator that outputs 200 Mb/sec.
\item Compare with the \href{https://en.wikipedia.org/wiki/Xorshift}{Xorshift} class of software pseudorandom number generators, which outputs (200 million 32-bit ints)/sec = 6.4 Gb/sec (\href{https://scholar.google.ca/scholar?hl=en&as_sdt=0\%2C5&q=xorshift&btnG=}{reference}) .
\end{itemize}
\end{frame}

\section{Excitation pulses in our simulator}

\begin{frame}
\begin{itemize}
\item To test my excitation routine, I initialized a simulation with 200 stationary ems evenly spaced along the $z$ axis between -10 m and 10 m. 
\item I applied a Gaussian-windowed sinc excitation pulse with duration 5.0 seconds, 10.0 seconds, and 20.0 seconds.
\item The transverse magnetization profile from the simulation (``Simulation") and from the small-tip-angle solution (``Theory") are shown in the following figures.
\item I am not sure why the transverse magnetization norms are different between simulation and theory, but there is agreement between the two on the spatial extent of the excitation.
\begin{itemize}
\item I think the error lies in my coding of the small-tip-angle solution. When I compute the tip angle at the origin by $\alpha = \int \gamma B_1(t) dt$ and compute $\mu_0 \sin(\alpha)$ I get agreement with the simulation ($\mu_0$ = equilibrium longitudinal magnetization).
\end{itemize}
\end{itemize}
\end{frame}

\begin{frame}{Pulse duration 5.0 seconds}
\begin{center}
\includegraphics[height=0.8\textheight]{{excitation_pulse-duration-5.0}.pdf}
\end{center}
\end{frame}

\begin{frame}{Pulse duration 10.0 seconds}
\begin{center}
\includegraphics[height=0.8\textheight]{{excitation_pulse-duration-10.0}.pdf}
\end{center}
\end{frame}

\begin{frame}{Pulse duration 20.0 seconds}
\begin{center}
\includegraphics[height=0.8\textheight]{{excitation_pulse-duration-20.0}.pdf}
\end{center}
\end{frame}

\end{document}