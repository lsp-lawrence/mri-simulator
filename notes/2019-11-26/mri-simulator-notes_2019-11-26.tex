\documentclass{beamer}
%\input{mybeamerdefs}
\usepackage{hyperref}
\hypersetup{colorlinks=true,
linkcolor=blue}
\newcommand{\real}{\mathbf{R}}
\DeclareMathOperator{\qvec}{\mathbf{vec}}
\DeclareMathOperator{\Diag}{\mathbf{Diag}}
\newcommand{\myemph}[1]{\emph{#1}}
\title{MRI Simulator Notes}
\author{Liam}
\date{\today}

\AtBeginSection[]{
  \begin{frame}
  \vfill
  \centering
  \begin{beamercolorbox}[sep=8pt,center,shadow=true,rounded=true]{title}
    \usebeamerfont{title}\insertsectionhead\par%
  \end{beamercolorbox}
  \vfill
  \end{frame}
}

\begin{document}

\begin{frame}
\maketitle
\end{frame}

\begin{frame}{Tasks}
\begin{itemize}
\item I looked into the use of quaternions to represent rotations in 3D space.
\item I looked into the means by which two different MRI simulators, SIMRI and JEMRIS, implement a numerical solution to the Bloch equation.
\end{itemize}
\end{frame}

\section{Quaternions}

\begin{frame}{Quaternions}
\begin{itemize}
\item A quaternion $q$ is a number that can be written in the form
\begin{equation*}
q = a + bi + cj + dk\,,
\end{equation*}
where $a,b,c,d$ are real numbers and $i,j,k$ are the imaginary units.
\item The real part of $q$ is the number $a$ and the vector or imaginary part of $q$ is the triple $(b,c,d)$.
\item The imaginary units follow rules for multiplication $i^2 = j^2 = k^2 = ijk = -1$.
\item Multiplication of two quaternions is associative but not commutative.
\end{itemize}
\end{frame}

\begin{frame}{Orientation and Rotation Quaternions}
\begin{itemize}
\item Quaternions give a compact way to represent orientations and rotations in 3D space.
\item Let $(p_1,p_2,p_3) \in \real^3$ be a position vector representing the orientation of an object in 3D space. The corresponding orientation quaternion is
\begin{equation*}
p = p_1i+p_2j+p_3k\,.
\end{equation*}
\item Let $(u_1,u_2,u_3) \in \real^3$ be a unit vector denoting an axis of rotation let and $\theta \in [0,2\pi]$ be an angle of rotation. The corresponding rotation quaternion is 
\begin{equation*}
\begin{aligned}
q &= \exp(\theta/2(u_1i+u_2j+u_3k))\\
& = \cos(\theta/2) + (u_1i+u_2j+u_3k)\sin(\theta/2)\,.
\end{aligned}
\end{equation*}
The latter expression follows from a generalization of Euler's formula.
\end{itemize}
\end{frame}

\begin{frame}{Orientation and Rotation Quaternions}
\begin{itemize}
\item The rotation of $(p_1,p_2,p_3)$ about the axis $(u_1,u_2,u_3)$ by angle $\theta$ yields the vector $(p_1',p_2',p_3')$, which may be computed using quaternions as follows:
\begin{equation*}
(p_1',p_2',p_3') = \qvec\lbrace qpq^{-1} \rbrace\,.
\end{equation*}
\item The rotation is clockwise if viewed along the direction of $(u_1,u_2,u_3)$.
\item The composition of rotations $q_1$ and $q_2$ may be represented by the single rotation quaternion $q_2q_1$.
\item \href{https://en.wikipedia.org/wiki/Quaternions_and_spatial_rotation}{reference}
\end{itemize}
\end{frame}

\begin{frame}{Quaternions versus Rotation Matrices}
\begin{itemize}
\item The composition of rotations using quaternions requires \myemph{fewer} computations than the composition of rotations using rotation matrices.
\item Rotating a vector using a quaternion requires \myemph{more} computations than rotating a vector using a rotation matrix.
\item \href{https://people.csail.mit.edu/bkph/articles/Quaternions.pdf}{reference}
\end{itemize}
\end{frame}

\begin{frame}{My thoughts on quaternions}
\begin{itemize}
\item Is it more computationally efficient to use quaternions or rotation matrices for our application? It's not clear to me which offers the advantage.
\end{itemize}
\end{frame}

%\section{Bloch Equation Simulation}
%
%\begin{frame}{Bloch Equation}
%\begin{itemize}
%\item Represent the magnetic moment of an em by $\boldsymbol{\mu} = (\mu_x,\mu_y,\mu_z) \in \real^3$. 
%\item Let the equilibrium longitudinal magnetic moment be $\mu_0$.
%\item We wish to solve the Bloch equation
%\begin{equation*}
%\frac{d\boldsymbol{\mu}}{dt} = \boldsymbol{\mu} \times \gamma \mathbf{B} - \frac{(\mu_z-\mu_0)}{T_1}\mathbf{z} - \frac{(\mu_x\mathbf{x}+\mu_y\mathbf{y})}{T_2}
%\end{equation*}
%numerically. The Bloch equation is a system of ODEs, hence we solve it by taking time steps.
%\item Over the time step $\Delta t$, three processes occur:
%\begin{enumerate}
%\item relaxation $\Rightarrow$ $\mu_x$ and $\mu_y$ diminish, $\mu_z$ grows.
%\item precession $\Rightarrow$ $\boldsymbol{\mu}$ rotates about the $z$ axis.
%\item excitation $\Rightarrow$ $\boldsymbol{\mu}$ rotates about an axis in the transverse plane.
%\end{enumerate}
%\item We simulate in a reference frame rotating about $z$ at the excitation frequency; on-resonance ems do not precess.
%\end{itemize}
%\end{frame}
%
%\begin{frame}{Relaxation}
%\begin{itemize}
%\item Relaxation is captured by
%\begin{equation*}
%\boldsymbol{\mu}(t+\Delta t) = A\boldsymbol{\mu}(t)+b\,,
%\end{equation*}
%where
%\begin{equation*}
%A = \Diag(\exp(-\Delta t/T_2),\exp(-\Delta t/T_2),\exp(-\Delta t/T_1))
%\end{equation*}
%and
%\begin{equation*}
%b = \qvec(0,0,M_0(1-\exp(-\Delta t/T_1)))\,.
%\end{equation*}
%\end{itemize}
%\end{frame}
%
%\begin{frame}{Precession}
%\begin{itemize}
%\item Precession is captured by
%\begin{equation*}
%\boldsymbol{\mu}(t+\Delta t) = R_z(\omega_z\Delta t)\boldsymbol{\mu}\,.
%\end{equation*}
%where $R_z$ is the rotation matrix about the $z$ axis.
%\item $\omega_z$ is the precession frequency offset of the em in the $z$ direction over the time step (frequency relative to the rotating frame frequency).
%\end{itemize}
%\end{frame}
%
%\begin{frame}{Free Precession}
%\begin{itemize}
%\item Free precession is captured by combining precession and relaxation.
%\item The relaxation matrix $A$ and the rotation matrix $R_z$ commute; hence, relaxation and precession can be applied in either order.
%\item Free precession is captured by
%\begin{equation*}
%\boldsymbol{\mu}(t+\Delta t) = AR_z(\omega_z\Delta t)\boldsymbol{\mu}(t)+b\,.
%\end{equation*}
%\end{itemize}
%\end{frame}
%
%\begin{frame}{Excitation}
%
%\end{frame}
%
%\begin{frame}{My Remaining Questions}
%\begin{itemize}
%\item $T_1$ and $T_2$ vary as a function of position. If an em moves between two positions of different $T_1$/$T_2$ over a time step, which value do we use?
%\item $B_0$ varies as a function of position and time. 
%\end{itemize}
%\end{frame}
%
%\begin{frame}{References}
%\begin{itemize}
%\item \href{http://mrsrl.stanford.edu/~brian/bloch/}{Brian Hargreaves from Stanford}
%\end{itemize}
%\end{frame}


\section{Numerical Solution of the Bloch Equation}

\begin{frame}{Bloch Equation}
\begin{itemize}
\item Represent the magnetization by $\mathbf{m} = m_x\mathbf{x}+m_y\mathbf{y}+m_z\mathbf{z}$.
\item Let the equilibrium longitudinal magnetization be $m_0$.
\item We wish to solve the Bloch equation
\begin{equation*}
\frac{d\mathbf{m}}{dt} = \mathbf{m} \times \gamma \mathbf{B} - \frac{(m_z-m_0)}{T_1}\mathbf{z} - \frac{(m_x\mathbf{x}+m_y\mathbf{y})}{T_2}
\end{equation*}
numerically.
\end{itemize}
\end{frame}

\begin{frame}{SIMRI}
\begin{itemize}
\item SIMRI is an MRI simulator from 2005.
\item SIMRI stores $\mathbf{m}(\mathbf{r},t)$ on a grid.
\item The magnetization vector is updated according to
\begin{equation*}
\mathbf{m}(\mathbf{r},t+\Delta t) = Rot_z(\theta_g)Rot_z(\theta_i)R_{\rm relax}R_{\rm RF}\mathbf{m}(\mathbf{r},t)\,.
\end{equation*}
\item $\theta_g$ and $\theta_i$ are precession angle changes due to the applied gradients and field inhomogeneities.
\item $R_{\rm relax}$ captures relaxation effects on the magnitude of $m_x,m_y,m_z$.
\item $R_{\rm RF}$ is a rotation by a specified flip angle about a specified axis in the transverse plane due to an RF pulse.
\item \href{https://www.ncbi.nlm.nih.gov/pubmed/15705518}{reference}
\end{itemize}
\end{frame}

\begin{frame}{My thoughts on SIMRI}
\begin{itemize}
\item Advantage: Bloch equation is implemented using matrix multiplication alone.
\item It might be more efficient to implement $Rot_z(\theta_g)Rot_z(\theta_i)R_{\rm RF}$ using quaternions.
\item Disadvantage: Cannot simulate arbitrary excitations, so ``selective excitation cannot be studied [with the SIMRI approach]" \href{https://www.ncbi.nlm.nih.gov/pubmed/20577987}{reference}.
\end{itemize}
\end{frame}


\begin{frame}{JEMRIS}
\begin{itemize}
\item JEMRIS stores the magnetization $\mathbf{m}(\mathbf{r},t)$ on a grid.
\item JEMRIS solves the Bloch equation in cylindrical coordinates using the ODE solver \href{https://computing.llnl.gov/projects/sundials/cvode}{CVODE}.
\item The magnetic field $\mathbf{B}(\mathbf{r},t)$ can be an arbitrary function of position and time.
\item Each position and time is associated with a set of physical parameters including $m_0$, $T_1$, $T_2$, $T_2^\star$.
\item \href{https://www.ncbi.nlm.nih.gov/pubmed/20577987}{reference}
\end{itemize}
\end{frame}

\begin{frame}{My thoughts on JEMRIS}
\begin{itemize}
\item Advantage: JEMRIS can simulate arbitrary excitations, so we can simulate selective excitation.
\item Disadvantage: Using the CVODE solver may be too computationally intensive for our purposes.
\end{itemize}
\end{frame}

\begin{frame}{We could use a hybrid approach?}
\begin{itemize}
\item Maybe we can simulate the excitation phase with the JEMRIS method and the readout phase with the SIMRI method?
\item Maybe we can simulate the excitation phase with the small-tip-angle solution to the Bloch equation? I'm not sure how this would be implemented, but it might be faster than the JEMRIS method.
\end{itemize}
\end{frame}

\end{document}