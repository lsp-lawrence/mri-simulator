\documentclass[dvipsnames]{beamer}
\input{mybeamerdefs}
\title{MRI Simulator Notes}
\author{Liam}
\date{\today}

\begin{document}

\begin{frame}
\maketitle
\end{frame}

\begin{frame}{Tasks}
\begin{itemize}
\item I ran the 2DFT simulation + reconstruction using single ems placed at different locations to look at the point-spread function.
\item I rewrote some pulse sequence functions to incorporate a user-specified TR and TE and ran the 2DFT simulation + reconstruction using the parameters Chuck suggested.
\item I looked at the effect on the total memory usage of increasing the number of ems and the pulse sequence length.
\item I modified the code so the time step between samples of the pulse sequence (rotations of the ems's magnetizations) is different from the time step for the ems's motion, metabolic conversion, etc.
\end{itemize}
\end{frame}

\section{2DFT point-spread function}

\begin{frame}{Summary}
\begin{itemize}
\item All the same simulation parameters as in 2019-12-10, which I found to be T1 = 10 ms, T2 = 10 ms, TE = 1 ms, TR = 53 ms.
\item I wanted to look at the ``asymmetric blurring" I noted in 2019-12-10. Chuck hypothesized it was due to the sum of multiple point-spread functions.
\item I placed an em with an offset in $x$, then the negative of the offset, and the same for $y$, to see if the point-spread function is simply reflected about the origin for these reflected offsets.
\end{itemize}
\end{frame}

\begin{frame}
\begin{center}
\includegraphics[width=\textwidth]{{reconstruction_x-0.5_y-0.0}.pdf}
\end{center}
\end{frame}

\begin{frame}
\begin{center}
\includegraphics[width=\textwidth]{{reconstruction_x-minus-0.5_y-0.0}.pdf}
\end{center}
\end{frame}

\begin{frame}
\begin{center}
\includegraphics[width=\textwidth]{{reconstruction_x-0.0_y-0.5}.pdf}
\end{center}
\end{frame}

\begin{frame}
\begin{center}
\includegraphics[width=\textwidth]{{reconstruction_x-0.0_y-minus-0.5}.pdf}
\end{center}
\end{frame}

\begin{frame}{Comments}
\begin{itemize}
\item The point-spread function itself appears to be asymmetric. Not sure what this means regarding the simulation and/or reconstruction. Is this typical in MR?
\item By eye, the point-spread function is reflected when the $x$ position is reflected; the same is untrue of $y$. $y$ is the phase-encoding direction, and the k-space lines are scanned starting with maximum positive $k_y$. Could this have something to do with it?
\end{itemize}
\end{frame}

\section{2DFT simulation with better parameters}

\begin{frame}{Summary}
\begin{itemize}
\item I redid the simulations of the previous section with the parameters T1 = 100 ms, T2 = 10 ms, TE = 4 ms, TR = 100 ms.
\end{itemize}
\end{frame}

\begin{frame}
\begin{center}
\includegraphics[width=\textwidth]{{reconstruction_T1-100ms_T2-10ms_TE-4ms_TR-100ms_x-0.5_y-0.0}.pdf}
\end{center}
\end{frame}

\begin{frame}
\begin{center}
\includegraphics[width=\textwidth]{{reconstruction_T1-100ms_T2-10ms_TE-4ms_TR-100ms_x-minus-0.5_y-0.0}.pdf}
\end{center}
\end{frame}

\begin{frame}
\begin{center}
\includegraphics[width=\textwidth]{{reconstruction_T1-100ms_T2-10ms_TE-4ms_TR-100ms_x-0.0_y-0.5}.pdf}
\end{center}
\end{frame}

\begin{frame}
\begin{center}
\includegraphics[width=\textwidth]{{reconstruction_T1-100ms_T2-10ms_TE-4ms_TR-100ms_x-0.0_y-minus-0.5}.pdf}
\end{center}
\end{frame}

\begin{frame}{Comments}
\begin{itemize}
\item The reconstructions are similar to those of the previous section.
\item There is more blurring for $(x,y) = (0.0,0.5)$ than in the previous section.
\item The point-spread function is reflected in $x$ but not in $y$ when the em position is reflected, again like the previous section.
\end{itemize}
\end{frame}

\section{Memory usage}

\begin{frame}{Summary}
\begin{itemize}
\item I instantiated a simulation object \texttt{Sim} with a variable number of ems (did not run simulation). I recorded the memory usage of this instantiation command. The plots on the next slides show the results.
\item I ran the simulation with a fixed number of ems and varied the pulse length. To save time, I took a single time step for each pulse. I recorded the memory usage of the \texttt{run\_sim} command. Changing the pulse length only changes the memory usage of the Pulse instantiation command, not the \texttt{Sim} instantiation command nor the \texttt{run\_sim} command.
\end{itemize}
\end{frame}

\begin{frame}{Liam's Macbook}
\begin{center}
\includegraphics[height=0.8\textheight]{mem_num_ems}
1.12 kB per em Macbook, 1.27 kB per em Supershop
\end{center}
\end{frame}

\begin{frame}{Comments}
\begin{itemize}
\item The memory usage per em is closer between the Macbook and Supershop as compared to before (0.6 to 1.2 kB per em), but still not identical. I do not know why. I checked that my Macbook is 64 bit.
\item Since changing the length of the pulse sequence only changes the memory usage of the pulse sequence object, I think we can infer that Python is pointing to a single instance of the pulse sequence and is not creating copies for each em.
\end{itemize}
\end{frame}

\end{document}