\documentclass[dvipsnames]{beamer}
\input{mybeamerdefs}
\title{MRI Simulator Notes}
\author{Liam}
\date{\today}

\begin{document}

\begin{frame}
\maketitle
\end{frame}

\begin{frame}{Tasks}
\begin{itemize}
\item I wrote a function to generate a 2DFT pulse sequence.
\item I wrote a function to reorder the matrix collected from the 2DFT sequence to an order suitable for calling the inverse DFT for image reconstruction.
\item I called the simulation and checked the activity monitor.
\item I started to write the script for simulating the 2DFT acquisition.
\end{itemize}
\end{frame}

\section{2DFT pulse sequence}

\begin{frame}{Summary}
\begin{itemize}
\item See the figure on the next slide.
\item The left column shows the excitation pulse.
\item The right column shows the set of k-space sampling pulses. The only difference between the sampling pulses is the amplitude of $G_y$ to change the line height in k-space.
\end{itemize}
\end{frame}

\begin{frame}{2DFT pulse sequence example}
\begin{center}
\includegraphics[height=0.8\textheight]{{pulse_sequence_2dft}}
\end{center}
\end{frame}

\begin{frame}{Comments}
\begin{itemize}
\item The basic shape of each waveforms looks as expected.
\item I am not sure what realistic values are for the various parameters -- maybe this is something we can discuss?
\end{itemize}
\end{frame}

\section{Matrix reordering function}

\begin{frame}{Summary}
\begin{itemize}
\item I computed the DFT of
\begin{equation*}
f(m,n) = 1 + \cos((2\pi/N)m) + \cos((2\pi/N)n)
\end{equation*}
with $N = 11$ and $m,n = 0,\ldots,N-1$ using negative frequencies to generate the matrix $S\_{acquired}$ (DC value at centre of matrix, etc.)
\item I computed $S\_{dft}$ from $S\_{acquired}$ using the \texttt{shift\_2DFT} function I wrote.
\item I compared the image generated this way to $f(m,n)$ (image from theory).
\end{itemize}
\end{frame}

\begin{frame}{Test results}
\begin{center}
\includegraphics[height = 0.8\textheight]{{shift_2DFT_test}}
\end{center}
\end{frame}

\section{Activity monitor}

\begin{frame}{Summary}
\begin{itemize}
\item My MacBook has 2 cores.
\item I ran the excitation simulation from 2019-12-04 and checked the activity monitor.
\item The Python CPU usage exceeds 100\%.
\end{itemize}
\end{frame}

\begin{frame}{Screenshot}
\begin{center}
\includegraphics[width=\textwidth]{{activity_monitor}}
\end{center}
\end{frame}

\section{Applying relaxation during excitation}

\begin{frame}{Summary}
\begin{itemize}
\item Right now, the excitation portion of the simulation represents the effect of each sample of the RF pulse by a rotation quaternion.
\item The effect of the entire RF pulse is computed by multiplying all these rotation quaternions together. This is more efficient than multiplying all of the corresponding rotation matrices together.
\item \href{https://computergraphics.stackexchange.com/questions/138/when-should-quaternions-be-used-to-represent-rotation-and-scaling-in-3d}{This post} on Stack Overflow seems to imply that you cannot represent non-uniform scaling (e.g. transverse and longitudinal relaxation) with quaternions; you need to use matrices.
\item We can either continue to neglect relaxation during excitation or else I can implement excitation using matrices and incorporate relaxation (the former is more efficient; the latter is more true to the physics).
\end{itemize}
\end{frame}

\section{Script for 2DFT simulation}

\begin{frame}{Summary}
\begin{itemize}
\item I distributed ems according to a Gaussian curve in $x$ and $y$ as shown in the figure following (with $z = 0$).
\item I will try to image this distribution of ems with a 2DFT pulse sequence tomorrow.
\end{itemize}
\end{frame}

\begin{frame}
\begin{center}
\includegraphics[width=\textwidth]{{density}}
\end{center}
\end{frame}

\end{document}